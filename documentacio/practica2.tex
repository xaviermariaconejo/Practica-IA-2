\documentclass[11pt,a4paper]{article}
\usepackage[utf8]{inputenc}
\usepackage[english,catalan]{babel}
\usepackage[gen]{eurosym}

%\usepackage[defaultsans]{droidsans}
%\usepackage[defaultserif]{droidserif}
%\usepackage[T1]{fontenc}


\author{
  Delicado Alcántara, Luis
  \\
  Conejo Micó, Xavier
  \\
  Sanchez Ferreres, Josep
}
\title{\Huge {Intel·ligència artificial:}\\ \huge{- Sistemes basats en el coneixement -}}

\begin{document}

\begin{titlepage}
\clearpage\maketitle
\thispagestyle{empty}
\end{titlepage}

\clearpage

\tableofcontents

\newpage

\section{Anàlisi del problema}
Aquesta pràctica consisteix en resoldre un problema donat mitjançant un sistema bastat en el coneixement (\emph{SBC}). És primordial a l'hora de dissenyar un \emph{SBC} el fet de realitzar un anàlisi complet del domini del problema des del punt de vista de l'enginyeria del coneixement, ja que usualment, les metodologies clàssiques provinents de l'enginyeria del software no són del tot adequades per a aquesta tasca. Per a resoldre aquest problema hem seguit una versió simplificada d'un d'aquests esquema d'anàlisi i resolució del problema repartit en 5 parts: Identifiació, Conceptualització, Formalització, Implementació i Prova, tal com es proposava a l'enunciat de la pràctica.

Als apartats que venen a continuació s'expliquen les decisions que hem pres a l'hora de resoldre el problema.

%==========================================================
%                      1. IDENTIFICACIÓ
%==========================================================
\subsection{Identificació}
El primer punt a tractar a l'hora de resoldre qualsevol problema és una correcta identificació d'aquest, prenent les consideracions necessàries, acotant el domini i definint correctament quin objectiu es vol assolir.

Pel nostre cas, l'objectiu a assolir és la creació d'un \emph{SBC} que serveixi per a recomanar possibles viatges als clients d'una agència de viatges, donades les característiques dels clients, les seves restriccions i/o preferències. Tots els conceptes que hem escollit representar i una justificació pertinent es troben en els apartats de Conceptualització i Formalització.
\subsubsection{Definició formal de solució}
Definim més formalment una solució al problema com una llista ordenada de tuples \big \langle Ciutat,Dies\big \rangle{} on, per cada ciutat, es guarda una llista de llocs a visitar i un allotjament d'aquesta ciutat. A més, també cal donar, per cada dos ciutats en el recorregut quin mitjà de transport es farà servir.\\
Addicionalment s'ha de donar el preu\footnote{Tots els preus calculats són per persona. Hem considerat pertinent fer aquesta simplificació per no haver de tenir en compte els descomptes per grups grans a l'hora de generar la solució.} total del viatge a partir dels preus de totes les coses a realitzar. Tal com es definia a l'enunciat, la resposta del \emph{SBC} ha de ser dues solucions diferents, és a dir, dos recomanacions possibles del viatge.

Definim a continuació alguns dels conceptes emprats en la definició:

\begin{description}
	\item[Visita:] Definim una \emph{visita} (o \emph{activitat}) com un a tupla \big \langle Nom,Preu,Durada\big \rangle{}. Per cada ciutat la suma de les durades de les activitats no poden superar el tems d'estada a la ciutat. Per exemple (``Visita al London Eye \(Londres\), durada mig dia, preu 25\euro'').
	\item[Allotjament:] Definim l'\emph{allotjament} anàlogament a la visita, però tenint en compte que la durada d'un allotjament coincideix amb el temps d'estada a la ciutat. Per exemple (``Hotel Juan Carlos I \(Barcelona\), preu per dia 150\euro'').
	\item[Mitjà de transport:] Un mitjà de transport es defineix entre un parell de ciutats, amb un nom i un identificador. També té un preu. Per exemple (``Vol: Paris-Londres, preu 200\euro'').
\end{description}

\subsubsection{Justificació de la complexitat del problema}
No costa gaire convencer-se que la complexitat del problema és bastant superior a quelcom resoluble mitjançant cerca heurística o local.

Veiem que d'entrada el criteri --o criteris-- a optimitzar no són del tot evidents i variaràn en funció del tipus de client. Per exemple, hi pot haver un client ric que vulgui un viatge de luxe sense importar el preu i també un client de classe mitja/baixa amb una clara restricció de preu. Podem repetir aquest argument pràcticament per tots els punts a considerar a l'hora de fer la solució. Resulta evident que condensar tota aquesta capacitat de decisió en una funció heurística resulta massa complicat com per intentar abordar el problema amb mètodes de cerca. Pensant en això també ens adonem que és primordial el coneixement d'un expert en el domini a l'hora de resoldre el problema, ja que en el tema de les recomanacions no existeix la optimalitat objectiva, i per tant la qualitat de les solucions dependrà del coneixement d'aquest expert i de com bo sigui recomanant viatges per tal de transferir el seu coneixement a l'\emph{SBC}. \\\\ Per tots aquests motius es pot justificar que un \emph{SBC} és una opció adecuada a l'hora de resoldre el problema.

\subsubsection{Identificació de les fonts del coneixement}
Tenint clar que cal resoldre el problema dissenyant un \emph{SBC}, el concepte de coneixement es torna primordial. En concret en aquest apartat justifiquem les fonts de coneixement del domini que ens permetràn dissenyar el sistema.

D'entrada, veiem clar que no disposem d'un expert en el domini del qual extreure el coneixement, és per això que haurem d'actuar nosaltres mateixos com experts en el tema fent servir el sentit comú a l'hora de prendre les decisions. No obstant, això no vol dir que no ens basem en coneixement real del domini, ja que els viatges són un domini on tothom té cert coneixement: ja sigui per experència pròpia o adquirida.

En un escenari més realista per una aplicació real identificariem clarament com a fonts de coneixement els empleats de l'agència de viatges, sobretot els que tinguin contacte directe amb el client, ja que són els que fan justament la feina de l'SBC. Per una altra banda també tenim el coneixement estàtic del món real (ciutats, hotels, etc.) que representarem a l'ontologia del domini.


%==========================================================
%                      2. CONCEPTUALITZACIO
%==========================================================
\subsection{Conceptualització}
Un cop ben definit quin és el problema a resoldre, cal plantejar-se quins son els conceptes importants del domini per tal de representar-los bé com a regles o bé com a coneixement estàtic a l'ontologia. També cal plantejar-se la divisió del problema gran en subproblemes menors que donin lloc a mòduls concrets d'una implementació posterior.

\subsubsection{Coneixement estàtic: La ontologia}

Un bon punt de partida és la solució del problema. D'allà ja veiem que els conceptes de ciutat, allotjament, mitjà de transport i activitat són necessaris per a l'\emph{SBC}. Aquests conceptes representen el coneixement estàtic del domini del que es disposa sense tenir en compte l'expert. Donada aquesta naturalesa de fets estàtics aquests conceptes seràn els que formaràn part de l'ontologia. Considerem que sense haver d'entrar en detalls d'implementació, la definició d'aquests conceptes ja s'ha fet prou detalladament a l'apartat anterior.

\subsubsection{Procés de resolució: Divisió en subproblemes}
\label{sec:subprob_informal}

Respecte el problema, hem identificat tres parts ben diferenciades: Determinar l'informació de l'usuari, raonar sobre aquella informació per tal de definir quins tipus de viatge són adecuats per ell i montar el viatge final. Més concretament definim aquests problemes com:

\begin{description}
	\item[Obtenir informació de l'usuari:] L'objectiu d'aquest problema és ben clar: Fer totes les preguntes necesàries a l'usuari per tal de classificar-lo en diversos tipus de perfil (nivell cultural, preferències, restriccions...) tot definint l'arquetip en el que col·loquem el client. És la part del problema que inclou una interacció directa amb l'usuari. 

	\item[Determinar el model de ciutat] Un cop sabem com és l'usuari, el següent pas és definir quins tipus de ciutats recomanarem a l'usuari, és a dir, construir un arquetip e solució.

	\item[Montar el viatge] L'últim pas és, un cop es té el model abstracte de solució, concretar-lo en un conjunt de ciutats, allotjaments i activitats en el format explicat a la definició de la solució.
\end{description}

En els apartats \ref{sec:formalitzacio} i \ref{sec:implementacio} es concreten els aspectes formals i tècnics d'aquesta subdivisió de problemes.


%==========================================================
%                      3. FORMALITZACIO
%==========================================================
\subsection{Formalització}
\label{sec:formalitzacio}
Tenint ja clars els conceptes que intervenen en el domini i sabent com s'ha subdividit el problema en subproblemes menors és el moment de formalitzar aquest coneixement. Cal, per una banda definir una ontologia\footnote{Veure apartat \ref{sec:ontologia} pels detalls sobre la ontologia, en aquest apartat ens centrarem en la descomposició en subproblemes} que representi el coneixement estàtic del domini com un conjunt de classes, atributs i relacions. Per altra banda és necessari definir formalment què ha de resoldre cadascun dels tres subproblemes i quines dades d'entrada i sortida tindrà cadascun d'ells per tal de facilitar l'implementació final.

Donem, a continuació, la definició formal dels subproblemes que hem utilitzat juntament amb una explicació de tots els punts pertinents. Per una definició més general de cadascun dels mòduls referir-se a la secció \ref{sec:subprob_informal} d'aquest mateix document.

\subsubsection*{Obtenir informació de l'usuari: Classificació}

Podem veure aquest mòdul com una cadena de preguntes que pretén classificar l'usuari en unes categories prefixades. Ja es pot parlar de coneixement expert en aquest mòdul, ja que tot i que s'obté pràcticament tota l'informació de l'usuari mitjançant preguntes, no totes les preguntes són necessàries en cada cas i s'obvien en els casos pertintents.

En aquest cas hem dividit el mòdul principal en dos submòduls. El primer es dedica exclusivament a determinar el tipus de client:

\begin{description}
	\item[Entrada:] Aquest mòdul no té dependències d'altres mòduls i la seva única font d'informació és l'entrada de l'usuari. 

	\item[Sortida:] La sortida d'aquest mòdul es recull en tota una sèrie de camps obtinguts directament o deduïts a partir de les preguntes que es fan al client.
	\begin{itemize}
	\item \emph{Tipus de viatge}. Possibles valors: Cultural, Diversio, Negocis, Romantic, Relax, Aventura.
	\item \emph{Mitjana d'edat}. Possibles valors: Jove, Adult, Jubilats, Variada.
	\item \emph{Tipus client}. Possibles valors: Grup Organitzat, Grup Amics, Institut, Familia, Parella, Individual, Companys de feina.
	\end{itemize}

\end{description}

\subsubsection*{Obtenir informació de l'usuari: Restriccions i preferències}

El segon mòdul està orientat a obtenir les preferències i/o restriccions que l'usuari vulgui imposar sobre el viatge.

\begin{description}
	\item[Entrada:] En aquest cas l'entrada correspon també només a l'entrada de l'usuari més alguns camps addicionals del submòdul anterior per tal de obviar algunes pregunes segons calgui.

	\item[Sortida:] La sortida d'aquest mòdul es recull en tota una sèrie de camps obtinguts directament o deduïts a partir de les preguntes que es fan al client. Addicionalment, i com una excepcíó, es classifica l'usuari per nivell de vida en aquest mòdul ja que és on es pregunta pel pressupost. Seria argumentable el posar-lo en el mòdul anterior i en tot cas hem optat per aquesta opció per tal de simplificar la gestió dels mòduls.
	\begin{itemize}
	\item \emph{Pressupost}. Dos enters, mínim i màxim d'euros.
	\item \emph{Duració}. Dos enters, mínim i màxim de dies.
	\item \emph{Nombre de ciutats} Dos enters, mínim i màxim de ciutats a visitar.
	\item \emph{Prefereix ciutats exòtiques}. Un predicat simple, si és cert vol dir que l'usuari prefereix centrar el seu viatge en ciutats exòtiques.
	\item \emph{Nivell de vida}. Possibles valors: Molt alt, Alt, Mig, Baix.
	\end{itemize}

\end{description}


\subsubsection*{Determinar el model de ciutat}

El tercer mòdul està orientat a obtenir, a partir del primer mòdul i algunes preferències de l'usuari, un model de ciutat que li interessi.

\begin{description}
	\item[Entrada:] En aquest cas l'entrada correspon a la descripció obtinguda sobretot en el primer mòdul juntament amb la informació que s'obté en el mòdul de restriccions i preferències. 

	\item[Sortida:] La sortida d'aquest mòdul es recull en tres camps obtinguts a partir de la informació de l'usuari. 
	\begin{itemize}
	\item \emph{Tipus de ciutat}. Pot tenir tres valors: molt coneguda, coneguda i desconeguda.
	Aquest camp es dedueix a partir del tipus d'activitat que s'ha obtingut en el primer mòdul juntament amb el tipus de client i si és un viatge exòtic.
	\item \emph{Cualitat Hotels}. Pot tenir tres valors: alt, mitja i baix.
	Aquest camp depèn del nivell de vida de l'usuari i en alguns casos del pressupost i els dies que van de viatge.
	\item \emph{Tipus d'activitats} Possibles valors: Cultural, Diversio, Negocis, Romantic, Relax, Aventura.
	
	\end{itemize}

\end{description}


\subsubsection*{Montar el viatge}



%==========================================================
%                      4. IMPLEMENTACIO
%==========================================================
\subsection{Implementació}
\label{sec:implementacio}
En aquesta secció es detallen les diverses decisions d'implementació que hem pres a l'hora de resoldre el problema plantejat. Cal notar que una explicació més completa sobre l'ontologia del domini es troba a l'apartat \ref{sec:ontologia}  

\clearpage


%==========================================================
%                5. DOCUMENTACIÓ ONTOLOGIA
%==========================================================
\section{Documentació de l'ontologia}%mirar si ontologia s'escriu amb l'
\label{sec:ontologia}
En aquesta secció explicarem com s'ha construït l'ontologia i analitzarem cadascuna de les classes que la componen. Per analitzar-les veurem per quins atributs estan compostes i quines relacions tenen amb les altres classes. També explicarem el motiu pel qual s'ha creat cadascuna de les classes.

Per explicar l'ontologia el primer que s'ha de comentar és que aquesta està dividida en dues parts diferenciables. Encara que siguin dues parts les classes tenen relació entre elles.

\subsection{Informació general}
Aquesta és una part constant que conté la informació que proporciona l'expert i que no té res a veure amb l'usuari. En aquesta part està formada per quatre classes.

\begin{description}
\item[Ciutat:]Aquesta classe conte dos atributs, el nom que és un String per identificar-les i la importància, que és un symbol, que pot tenir els valors molt coneguda, coneguda i desconeguda, ja que, depenent de l'usuari voldrà anar a un tipus o un altre.
A part, té una relació amb la classe activitat i un altre amb allotjament ambdós amb un multislot. Aquests serveix per guardar quines són les activitats i l'allotjaments que ofereix cada ciutat i així saber que si una persona va a una ciutat quines activitats pot fer i a quins hotels es poden allotjar.\\
Un altre aspecte d'aquesta classe és que conté cinc subclasses, que algunes d'aquestes també tenen subclasses, i això ens ajuda a l'hora de crear els vols, ja que, si són d'un altra zona, el preu serà més elevat que si són de la mateixa.

\item[Allotjament:] Aquesta classe conte dos atributs, el nom de l'allotjament, en un String, per poder identificar-ho i un altre atribut que és el preu de l'allotjament per dia i persona, que es guarda com un enter, que ens serveix per calcular els diners que es gastarà un usuari.\\
Aquesta classe també té cinc subclasses que s'utilitzen per identificar la qualitat de l'hotel depenent de quantes estrelles té i així, depenent de quin tipus d'usuari demana un viatge, s'ofereix un tipus o un altre.

\item[Activitat:] Aquesta classe té tres atributs, el primer, com a les classes anteriors un nom per identificar l'activitat, després tenim el preu que costa fer l'activitat amb un enter, i per últim la durada de l'activitat, que es guarda com un symbol amb dos valors possibles, migdia i un dia, i així es pot saber quantes activitats com a màxim poden fer en el temps que estigui en una ciutat.\\
També conte unes subclasses per identificar el tipus de l'activitat i així escollir activitats al gust de l'usuari.


\item[Viatge:] Aquesta classe conte dos atributs, el primer, com en les dues ultimes classes, és un enter que s'utilitza per guardar el preu del viatge, el segon, es guarda com un symbol que identifica el tipus de transport que utilitza aquell viatge i pot tenir quatre valors: avio, tren, autobús i vaixell.
També conte una doble relació amb la classe ciutat. Aquesta relació ens dóna la informació de quin és l'origen i el destí del viatge.
\end{description}

\subsection{Solució} 
Després tenim una part que, com el títol diu, es guarda la informació final que compon un viatge. En aquesta part està formada per dues classes.

\begin{description}
\item[Solució parcial:] Una solució parcial la podem definir com l'etapa en la qual l'usuari passa en una ciutat del viatge. Amb aquesta definició podem deduir que tindrà tres relacions amb tres classes diferents.
Primer tenim la ciutat a la qual correspon aquesta solució parcial, després l'hotel de la ciutat en el que s'allotja i per últim un conjunt d'activitats que són les que es faran.
Després tenim dos atributs, el primer és el preu que costa fer tota la solució parcial que es guarda en un enter i després tenim la durada d'aquest que es guarda en un altre enter i ens facilita el càlcul total del temps i el preu de tot el viatge.

\item[Solució final:] Una instancia de solució final es defineix com un viatge complet i, per tant, té una relació amb la classe solució parcial amb un multslot. Aquesta relació és la que guarda totes les estades de l'usuari per les ciutats que visitarà juntament amb les activitats. També té una altra relació amb la classe viatge per saber com va d'una de les ciutats de la relació anterior a la següent. Per últim, conte un atribut que conté el preu final del viatge contant els preus dels viatges i de les estades a cada ciutat.

\end{description}

\clearpage

%==========================================================
%                    6. JOCS DE PROVES
%==========================================================
\section{Jocs de proves}
En aquesta secció presentem diversos jocs de proves generats per tal d'evaluar el rendiment i qualitat del nostre \emph{SBC} tot comentant els resultats obtinguts. Per cadascuna de les proves hem inclòs els objectius, comportament esperat, resultat de la prova. Al final de cada prova s'inclouen comentaris sobre els resultats.

\subsection{Un viatge romàntic de parella}
Per aquesta prova demanarem una recomanació de viatge d'una parella jove a l'\emph{SBC}.\\

\noindent
\begin{tabular}{|p{0.28\textwidth}|p{0.73\textwidth}|}
\hline
\textbf{\mbox{Condicions de la} \mbox{prova}} & Cas típic de viatge de parella de caire romàntic per diverses ciutats poc conegudes. Entre 3 i 5 dies i per menys de 800\euro{} per persona.\\
\hline
\textbf{Comportament \mbox{esperat de l'\emph{SBC}}} & Donada la naturalesa del viatge demanat esperem que l'SBC esculli un conjunt de ciutats exòtiques arreu del mon. No obstant donada la forta restricció de preu és d'esperar que s'escullin activitats de baix cost i hotels més aviat barats. Cal comentar també que s'espera que les activitats siguin de caire romàntic, donat que és la naturalesa principal del viatge. \\
\hline
\textbf{\mbox{Resultats de la} \mbox{prova}} & \\
\hline
\end{tabular}

Aenean tristique fermentum mauris, a pharetra ex varius rutrum. Aliquam consequat faucibus enim eget ullamcorper. Sed sapien odio, convallis et risus ac, condimentum tincidunt diam. Phasellus rutrum ullamcorper sem, id finibus lectus maximus sed. Nunc non ornare est, quis sodales ligula. Duis laoreet nisl vel magna auctor imperdiet. Maecenas ac magna vitae enim facilisis pharetra nec nec magna.

\end{document}
