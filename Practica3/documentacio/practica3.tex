\documentclass[11pt,a4paper]{article}
\usepackage[utf8]{inputenc}
\usepackage[english,catalan]{babel}
\usepackage[gen]{eurosym}
\usepackage[pdftex, pdfborderstyle={/S/U/W 0}]{hyperref} % this disables the boxes around links
\usepackage{longtable}

\usepackage{listings}
\usepackage{color}

\definecolor{dkgreen}{rgb}{0,0.6,0}
\definecolor{gray}{rgb}{0.5,0.5,0.5}
\definecolor{mauve}{rgb}{0.58,0,0.82}

\lstset{frame=tb,
  language=Java,
  aboveskip=3mm,
  belowskip=3mm,
  showstringspaces=false,
  columns=flexible,
  basicstyle={\small\ttfamily},
  numbers=none,
  numberstyle=\tiny\color{gray},
  keywordstyle=\color{blue},
  commentstyle=\color{dkgreen},
  stringstyle=\color{mauve},
  breaklines=true,
  breakatwhitespace=true,
  tabsize=4,
  extendedchars=true,
  literate={á}{{\'a}}1 {à}{{\`a}}1 {é}{{\'e}}1 {ó}{{\'o}}1
}

\author{
  Delicado Alcántara, Luis
  \\
  Conejo Micó, Xavier
  \\
  Sanchez Ferreres, Josep
}
\title{\Huge {Intel·ligència artificial:}\\ \huge{- Planificació -}}

\begin{document}

\begin{titlepage}
\clearpage\maketitle
\thispagestyle{empty}
\end{titlepage}

\clearpage

\tableofcontents

\newpage

\section{Model del domini}
\subsection{Variables}
\subsection{Predicats}
\subsection{Operadors}

\section{Model dels problemes a resoldre}
\subsection{Objectes}
\subsection{Estat inicial}
\subsection{Estat final}

\subsection{Generador d'instàncies}
\label{sec:script}

Com a complement al model del domini en PDDL, hem creat un script en bash que permet generar instàncies aleatòries del problema tenint en compte diversos paràmetres per acotar-ne la mida. L'objectiu d'aquest script és generar un arxiu .pddl complert i directament \emph{parsejable} per \emph{Metric-FF} amb una instància de problema associada a un dels dominis --o extensions--. Aquest script l'utilitzem posteriorment per a experimentar amb el temps d'execució del solver (veure apartat \ref{sec:experiment}).

Els paràmetres que rep l'script són:
\begin{itemize}
  \item{Nombre de ciutats total. L'anomenarem $N$. Les ciutats es generen amb els noms \emph{viatge-i} ($i \in \{1..N\}$) }
  \item{Nombre mínim de ciutats del viatge. L'anomenarem $C$.}
  \item{Nombre mínim de dies del viatge. L'anomenarem $D$.}
  \item{Nom de la instància del problema a generar.}
\end{itemize}

Totes les altres dades es generen aleatòriament. A continuació expliquem amb quina estratègia s'han generat totes les dades detalladament.

\subsubsection*{Ciutats}

L'script guarda els predicats referents a ciutats com un resultat intermig a l'arxiu \texttt{.tmp/ciutats.txt}

Per les ciutats el procediment és senzill. Es creen $N$ ciutats i es decideix aleatòriament dos valors consistents entre 1 i 5 pel màxim i mínim de dies que es pot estar a cadascuna de les ciutats\footnote{Teniem dubtes sobre si l'enunciat es referia a màxim i mínim individuals per cada ciutat o màxim i mínim únic per totes les ciutats. Després de pensar-hi ens va semblar més coherent l'última opció i ho vam implementar així}.

\subsubsection*{Hotels}

L'script guarda els predicats referents a hotels com un resultat intermig a l'arxiu \texttt{.tmp/hotels.txt}

Per a cada ciutat s'escull un nombre aleatori $h$ i es generen d'entre 1 a 4 hotels en aquella ciutat. Els noms dels hotels es generen consecutivament (hotel-1, hotel-2, ...), però com que cada ciutat té un nombre diferent d'hotels no és possible fer la conversió directa d'\emph{hotel-i} a \emph{ciutat-j} i cal mirar al fitxer resultant per saber en quina ciutat està cada hotel consultant els predicats pertinents.

El preu de cada hotel es decideix aleatòriament. Cal notar que el fet de crear diversos hotels amb preus diferents és simplement per provar si les extensions d'optimització son capaces de trobar l'hotel més barat de cada ciutat.

\subsubsection*{Vols}

L'script guarda els predicats referents a vols com un resultat intermig a l'arxiu \texttt{.tmp/vols.txt}

Per cada parell de ciutats diferents $(i,j)$, amb probabilitat $fly\_factor$ l'script genera un vol de \emph{ciutat-i} a \emph{ciutat-j} amb un preu aleatori que intenta ser coherent amb el preu dels hotels. 

Per a generar el graf de vols, cal assegurar que aquest sigui connex. Per fer-ho, ens basem en un resultat obtingut en l'assignatura l'Algorísmia: ``Amb alta probabilitat, si s'escullen $n \ln(n)$ arestes en un graf, on $n=|V|$, aquest graf serà connex.''. Per tal que en mitja el nombre de vols s'aproximi a aquest $n \ln(n)$ calculem el $fly\_factor$ com:

\[
\ fly\_factor = \frac{N \ln(N)}{\frac{N (N-1)}{2}}
\]

Com que la probabilitat d'escollir una aresta és $fly\_factor$, en mitjana és evident que el nombre d'arestes escollides és $N\ln(N)$. Podriem argumentar que aquesta manera de generar-ho no és exactament la mateixa ja que es poden escollir menys arestes de les que s'ha calculat. Cal veure, però, que tampoc és un requisit que el graf sigui completament connex, només cal que contingui una component connexa K amb $|K| >= C$. És interessant, doncs, generar aquest possible entrebanc per l'algoritme i veure si és capaç de resoldre'l per tal de generar unes instàncies de problema menys trivials\footnote{Si detectéssim que un joc de proves és irresoluble en generariem un altre. Aquesta feina es podria automatitzar però hem escollit no fer-ho ja que en cap de les proves realitzades s'ha donat cap cas on el problema generat acabés no tenint solució}.

\subsubsection*{Generar el fitxer pddl}

Un cop creats els tres fitxers intermitjos, aquests s'ajunten en un fitxer pddl juntament amb tots els predicats i fluents necessaris que s'hagin d'inicialitzar per tal que funcioni el solver en el nostre domini. Hem posat especial cura en que el fitxer pddl de sortida es trobi ben tabulat i estructurat per tal de facilitar-ne la lectura si fós necessari.

Com que l'objectiu de l'script és generar el fitxer pddl complet, també s'inclou el goal pel problema, que simplement es troba \emph{hardcoded} dins l'script.

\section{Descripció de la metodologia de treball}


\section{Jocs de prova}
\subsection{Bàsic}
\subsection{Extensió 1}
\subsection{Extensió 2}
\subsection{Extensió 3}
\subsection{Extensió 4}
\subsection{Extensió 4}

\section[Experiment extra]{Experiment: Evolució del temps d'execució per problemes de mida creixent}
\label{sec:experiment}

En aquest apartat ens plantejem estudiar com evoluciona el temps d'execució del solver \emph{Metric-FF} per a instancies del nostre problema de mida creixent en la extensió 3. Per fer-ho, hem generat les instàncies amb l'script explicat en \ref{sec:script} lleugerament modificat\footnote{L'script genera instàncies de l'extensió 4 i crea predicats innecesaris}. En els següents apartats es descriuen els passos que hem seguit per realitzar l'experiment juntament amb els resultats obtinguts.

\subsection{Plantejament i hipòtesi}

En essència, el que s'està resolent és un problema d'optimització com el que es podria resoldre amb una cerca local. De fet, el solver que utilitzaem fa servir tècniques de cerca local i heurística. És raonable, doncs, plantejar-nos com evolucionaria el temps d'execució per a mides creixents del problema i estudiar el comportament del solver. 

Creiem que \emph{Metric-FF} tindrà un temps quadràtic (o com a molt cúbic) per a les instàncies del problema. No esperem trobar el comportament de greedy característic de Fast Forward en els resultats ja que en aquest cas s'està usant el mode d'optimització i per tant assumim que l'algoritme ha de fer una cerca forçosament més exhaustiva.

\subsection{Condicions de l'experiment}

Per a preparar l'experiment hem generat 60 jocs de prova amb el nostre script amb mides de 5 fins a 300 ciutats. 

Cal notar que per cadascuna de les $N$ escollides hem generat 3 jocs de prova diferents per tal de fer la mitjana entre aquests, i que mesurarem els temps d'execució 5 vegades per cadascun dels jocs de proves. Aquest procés el fem per tal de normalitzar els resultats i eliminar qualsevol biaix que un problema particular pugui introduïr en les dades.

\subsection{Resultats de l'experiment}



\subsection{Conclusions}


\end{document}